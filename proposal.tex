\documentclass[final]{article}
\usepackage[utf8]{inputenc} % allow utf-8 input
\usepackage[T1]{fontenc}    % use 8-bit T1 fonts
\usepackage{hyperref}       % hyperlinks
\usepackage{url}            % simple URL typesetting
\usepackage{booktabs}       % professional-quality tables
\usepackage{amsfonts}       % blackboard math symbols
\usepackage{nicefrac}       % compact symbols for 1/2, etc.
\usepackage{microtype}      % microtypography
\usepackage{xcolor}         % colors

\title{Project Proposal:  Planner for Family Activities}
\author{%
  Patiharn \& Liangkobkit \\
  1012871867, patiharn.liang@gmail.com \\
    \scriptsize{\url{https://colab.research.google.com/drive/1vh_D2th_JABNn6x3Yv3K8aDz92ni-TSg?usp=sharing}}
}

\begin{document}

\maketitle

\vspace{-0.5in}

\section*{Introduction}
This project develops a Google-style hybrid search planner that retrieves and ranks age-appropriate family activities. Combining BM25, Sentence-BERT embeddings, FAISS, and a neural re-ranker, the system personalizes results, respects calendar constraints, and exports weekly timetables, adapting modern deep learning retrieval methods to practical family scheduling.


\section*{Illustration}
\begin{figure}[h]
\centering
\fbox{\parbox{0.9\linewidth}{
Corpus $\rightarrow$ Hybrid Retrieval (BM25 + Sentence-BERT + FAISS + RRF) $\rightarrow$ Neural Re-ranker $\rightarrow$ Planner (Calendar Overlay) $\rightarrow$ Exports (ICS/PDF/CSV)
}}
\caption{System overview diagram.}
\end{figure}

\section*{Background \& Related Work}
We build on foundational methods in information retrieval and modern semantic search. \textbf{BM25} provides strong keyword-based baselines \emph{[Robertson \& Zaragoza, 2009]}. \textbf{Sentence-BERT} enables dense embeddings for semantic similarity \emph{[Reimers \& Gurevych, 2019]}. \textbf{FAISS} supports efficient approximate nearest-neighbor search at scale \emph{[Johnson et al., 2017]}. \textbf{MMR} introduces diversity-aware ranking \emph{[Carbonell \& Goldstein, 1998]}.  
For content, we collect activity data from trusted family and education resources: \emph{Raising Children Network}, \emph{Active for Life}, \emph{Oxford Owl}, \emph{Positive Action},  \emph{Escape Room Geeks} and more.

\section*{Data Processing}

\textbf{Corpus.} The dataset with more than 100 activities is gathered with the following columns:  
\texttt{title, age\_min, age\_max, duration\_mins, tags, cost, indoor\_outdoor, season, materials\_needed, how\_to\_play, players, \\ parent\_caution}.  \\
\textbf{Data Cleaning.} The text columns are turned to be list columns in the dataset, such as \texttt{materials\_needed} and \texttt{how\_to\_play}.



\section*{Architecture}
\textbf{Hybrid Search (Google-style, small scale).} The system combines \emph{BM25} (sparse) and \emph{Sentence-BERT} (dense) with \emph{FAISS} ANN; results are merged via \emph{Reciprocal Rank Fusion} (RRF) and finalized by a \emph{neural re-ranker}.

\textbf{Sparse Retrieval (BM25).} Keyword-based scoring over the corpus to capture exact intent and rare-term importance.

\textbf{Dense Retrieval (Sentence-BERT).} Sentence-level embeddings for semantic similarity so paraphrases/synonyms are retrieved even without exact term overlap.

\textbf{Approximate Nearest Neighbor (FAISS).} Efficient large-scale vector search for dense retrieval, enabling fast top-$K$ candidate generation.

\textbf{Fusion (RRF).} Combines BM25 and dense candidate lists into a single slate, trading off precision/recall robustly across queries.

\textbf{Neural Re-ranker (feed-forward / cross-encoder–style).} Reorders the fused top-$K$ using features such as \emph{AgeFit}, \emph{PreferenceMatch}, \emph{LoadBalance}, \emph{ContextFit}, \emph{Practicality}, plus diversity via \emph{MMR}.

\textbf{Planner \& Visualization.} Greedy slotter that respects calendar busy blocks, visualized on a calendar overlay and integrated with read-only Google Calendar data.

\textbf{Frontend.} Next.js interface with toggles (\emph{Exclude on Export}) and filters.

\textbf{Exports.} ICS (planned activities only, or including events from Google Calendar data), CSV, and PDF weekly grid.


\section*{Baseline Model}
The baseline is a \textbf{BM25-only ranker} with simple metadata filters (age, duration). It does not use semantic embeddings, personalization, or diversity constraints.

\section*{Ethical Considerations}
\textbf{Privacy.} Calendar integration is read-only; exports can exclude personal events.  \\
\textbf{Child Safety.} Activities are age-filtered, and flagged for supervision.  \\
\textbf{Bias.} We monitor cost and age balance; MMR ensures diversity and fair exposure.  

{\small
\begin{thebibliography}{9}\setlength{\itemsep}{2pt}

\bibitem{robertson2009bm25}
S.~E. Robertson and H. Zaragoza.
The probabilistic relevance framework: BM25 and beyond.
\emph{Foundations and Trends in IR}, 3(4):333--389, 2009.

\bibitem{reimers2019sentencebert}
N. Reimers and I. Gurevych.
Sentence-BERT: Sentence embeddings using Siamese BERT-networks.
In \emph{EMNLP}, 2019.

\bibitem{johnson2017faiss}
J. Johnson, M. Douze, H. Jégou.
Billion-scale similarity search with GPUs (FAISS).
\emph{arXiv:1702.08734}, 2017.

\bibitem{carbonell1998mmr}
J. Carbonell and J. Goldstein.
The use of MMR, diversity-based reranking for reordering documents and producing summaries.
In \emph{SIGIR}, 1998.

\bibitem{activeforlife57}
Active for Life.
57 fun physical activities to do with kids aged 2 to 4.
\url{https://activeforlife.com/57-fun-physical-activities-to-do-with-kids-aged-2-to-4/}, 2024.

\bibitem{activeforlife50}
Active for Life.
50 indoor physical activities for kids.
\url{https://activeforlife.com/50-indoor-physical-activities-for-kids/}, 2024.

\bibitem{oxfordowl}
Oxford Owl.
Fun learning activities for 6--7 year olds.
\url{https://home.oxfordowl.co.uk/kids-activities/learning-activities-age-6-7/}, 2025.

\bibitem{positiveaction}
Positive Action.
20 evidence-based social skills activities and games for kids.
\url{https://www.positiveaction.net/blog/social-skills-activities-and-games-for-kids}, 2024.

\bibitem{escaperoom}
Escape Room Geeks.
30 indoor games for kids to burn out energy at home.
\url{https://escaperoomgeeks.com/indoor-games-for-kids/}, 2024.

\end{thebibliography}
}

\end{document}
